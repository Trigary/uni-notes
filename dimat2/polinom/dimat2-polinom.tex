% !TeX spellcheck = hu_HU
\documentclass[12pt,a4paper]{article}
\usepackage[utf8]{inputenc}
\usepackage{cmap}
\usepackage[T1]{fontenc}
\usepackage[magyar]{babel}
\usepackage{amsmath}
\usepackage{amsfonts}
\usepackage{amssymb}
\usepackage{graphicx}

\usepackage{outlines}
\usepackage{hyperref}

\hyphenpenalty=10000

\begin{document}

\begin{center}
	\huge
	Diszkrét matematika II\\
	\vspace{1mm}
	\LARGE
	Polinomok témakör jegyzete\\
	\vspace{5mm}
	\large
	Készült Burcsi Péter előadásai\\
	és Harrison-Juhász Zsófia gyakorlatai alapján\\
	\vspace{5mm}
	Sárközi Gergő, 2021-22-1. félév\\
	Nincsen lektorálva!
\end{center}

\tableofcontents

\pagebreak

\section{Gyűrű, Test, integritási tartomány}

\subsection{Gyűrű ($R,+,*$)}

\begin{outline}
	\1 Példa: $\mathbb{Z}$, $\mathbb{Q}$, $\mathbb{R}$, $\mathbb{C}$, $\mathbb{Z}_m$ (mod m maradékosztályok gyűrűje)
	\1 $+$: R-en művelet
		\2 asszociatív, kommutatív
		\2 létezik nullelem: 0 ($a+0=0+a=a$)
		\2 minden elemnek van inverze ($a-a=0$)
	\1 $*$: R-en művelet
		\2 asszociatív
		\2 nem feltétlenül kommutatív (pl. mátrixok miatt: ott nem azok)
	\1 $*,+$: $*$ disztributív $+$-ra: $x*(y+z)=x*y+x*z$
\end{outline}

\subsection{Integritási tartomány}

\begin{outline}
	\1 Példa: $\mathbb{Z}$, $\mathbb{Q}$, $\mathbb{R}$, $\mathbb{C}$, $\mathbb{Z}_p$ ahol p egy prím
	\1 Egy R gyűrű integritási tartomány, ha:
		\2 legalább 2 eleme van
		\2 szorzás kommutatív
		\2 gyűrű (szorzás) nullosztómentes ($a*b=0 \implies a=0 \lor b=0$)
			\3 ellenpélda: $\mathbb{Z}_{10}$: $6*5=0$
\end{outline}

\subsection{Egységelemes Integritási Tartomány (EIT)}

\begin{outline}
	\1 Integritási tartomány, ahol létezik $1 \in R$: $a*1=1*a=a$
	\1 Van $+,*,-$: ahogy megszoktuk, pl.: $ax=ay \implies a=0 \lor x=y$
\end{outline}

\subsection{Test}

\begin{outline}
	\1 Gyakorlatilag: egy E.I.T. ahol van osztás
	\1 $\forall a \in R: a \ne 0 \implies \exists a^{-1}: a*a^{-1}=1$
	\1 pl.: $\mathbb{Q}$, $\mathbb{R}$, $\mathbb{Z}_p$ ahol $p$ prím
	\1 Minden $f$ polinom elosztható maradékosan bármely $g \ne 0$ polinommal
\end{outline}

\pagebreak

\section{Polinom}

\begin{outline}
	\1 Egy R gyűrű feletti polinomokon olyan $(f_0,f_1,f_2,...)$ végtelen sorozatokat értünk,
	ahol $f_j \in R$ és $\exists n \in \mathbb{N}: j>n \implies f_j=0$
	\1 Példa: $6+5x+2x^2+3x^4$ $\to$ $(6,5,2,0,3,0,0,0,...)$
	\1 R feletti polinomok halmaza: $R[x]$ (x: polinomok határozatlanja)
		\2 Ha R E.I.T., akkor $R[x]$ is E.I.T.
	\1 Polinomfüggvény: $f \in R \to R$, $f(r)=f_0+f_1*r+...+f_{deg(f)}*r^{deg(f)}$
\end{outline}

\subsection{Polinom foka}

\begin{outline}
	\1 polinom foka ($\deg f$): legnagyobb $n$ amelyre $f_n \ne 0$
		\2 a fok $-\infty$, ha $f=(0,0,0,...)$
	\1 főegyüttható: $f_n$ ahol $n=\deg f$
	\1 konstans tag: $f_0$
	\1 f lineáris, ha $\deg f \le 1$
	\1 f konstants, ha $\deg f \le 0$
\end{outline}

\subsection{Összeadás, szorzás $R[x]$-en}

\begin{outline}
	\1 $f=(f_0,f_1,f_2,...)$ és $g=(g_0,g_1,...)$
	\1 $f+g=(f_0+g_0,f_1+g_1,...)$
		\2 $deg (f+g) \le max(deg(f),\; deg(g))$
	\1 $f*g=h$ ahol: $h_k=\sum_{i+j=k}f_i*g_j$
		\2 Példa: $f=(2,0,1,3,0,...)$ és $g=(7,-2,3,0,...)$\\
		$f*g=(?,2*3+0*-2+1*7,...)=(?,f_0*g_2+f_1*g_1+f_2*g_0,...)$
		\2 $\deg (f*g) \le \deg(f)+\deg(g)$
			\3 egyenlő, ha $R$ egy integritási tartomány: $f_{\deg(f)}*g_{\deg(g)} \ne 0$
\end{outline}

\subsection{Maradékos osztás tétele}

\begin{outline}
	\1 Legyen $R$ egy E.I.T. és $f,g \in R[x]$ és $n=\deg f$ és $m = \deg g$
	\1 Ha $\exists {g_m}^{-1}$ (van reciproka a főegyütthatónak)
	\1 Akkor $\exists! g,r \in R[x]: f = g*q+r \wedge \deg r < \deg g$
	\1 Egyértelműség bizonyítása:
		\2 Legyen $f=g*q_1+r_1=g*q_2+r_2$
		\2 Ebből következik: $g*(q_1-q_2)=r_2-r_1$
		\2 Fokokra áttérve: $\deg g + \deg (q_1 - q_2) = \deg (r_2 - r_1)$
		\2 Ha $q_1 \ne q_2$: a bal oldal $\ge \deg g$, a jobb oldal $< \deg g$
		\2 Tehát $(q_1 = q_2) \implies (g*0=0) \implies (r_r - r_2 = 0)$
	\1 Létezés bizonyítása:
		\2 $q = f_n*g_m^{-1}*x^{n-m}+q^*$ \;\; (ahol $q^*$ ugyan ez $f^*$-gal)
		\2 $f^*=f-g*f_n*g_m^{-1}*x^{n-m} = g*q^*+r^*$
		\2 Rekurzió alapesete: ha $\deg f < \deg g$ akkor $q=0$ és $r=f$
		\2 $f$ fokszáma csökken, $\deg f^* < \deg f$ belátása:\\
		$f^*_n=f_n*x^n-g*f_n*g_m^{-1}*x^{n-m} = f_n*x^n-f_n*x^n = 0$
\end{outline}

\subsection{Polinom osztás a gyakorlatban}

\subsubsection{Bizonyításhoz hasonlóan, rekurzívan}

\begin{outline}
	\1 rekurzívan $f$-ből mindig kivonjuk $g*f_n*g_m^{-1}*x^{n-m}$-ot
	\1 megállunk, ha $\deg f < \deg g$, ekkor $r=f$
	\1 $q=\sum f_n*g_m^{-1}*x^{n-m}$
\end{outline}

\subsubsection{Horner táblázat}

\begin{outline}
	\1 $f = (x-c)*q + r$ ahol $\deg r \le 0$ \;\; (azaz $g=x-c$ és $\deg g = 1$)
\end{outline}

\begin{table}[h]
	\centering
	\begin{tabular}{|c|c|c|c|c|c|c|}
		\hline
		& $f_n$ & $f_{n-1}$ & $f_{n-2}$ & ... & $f_0$ & \\
		\hline
		c & $\times$ & $c_1=f_n$ & $c_2=c_1*c+f_{n-1}$ & .. & $c_n=c_{n-1}*c+f_1$ & $c_{n+1}=...=f(c)$ \\
		\hline
	\end{tabular}
\end{table}

\pagebreak

\section{Gyöktényező kiemelése, következményei}

\subsection{Gyöktényező kiemelése}

\begin{outline}
	\1 Ha $R$ egy E.I.T. és $f \in R[x]$ és $f(c)=0$
	\1 Akkor $\exists q \in R[x]: f = q*(x-c)$
	\1 Bizonyítás:
		\2 $f$ osztása $x-c$-vel: $f = q*(x-c)+r$
		\2 $\deg r < \deg x-c = 1 \implies r$ konstans
		\2 $c$ gyök $\implies 0=f(c)=q*(c-c)+r \implies r=0$
\end{outline}

\subsection{Gyökök száma max a polinom fokszáma}

\begin{outline}
	\1 Legyen $R$ egy E.I.T. (!!!) és $f \in R[x]$ és $\deg f = n$
	\1 Ekkor $f$-nek max $n$ gyöke van
	\1 Bizonyítás: (indukcióval)
		\2 $n=0 \implies f(x)=r \implies $ nincs gyök ha $r \ne 0$
			\3 ha $r = 0$ akkor $f=0$ azaz $\deg f = -\infty$
		\2 indukciós lépés: $\exists c$ gyök $\implies f=(x-c)*q$
			\3 $x-c$ az 1 gyök
			\3 $\deg q = n-1$, indukciós feltevés alapján max $n-1$ gyöke van
			\3 $1+$ max $n-1=$ max $n$, tehát készen vagyunk
\end{outline}

\subsection{Polinomok egyenlősége több behelyettesítés alapján}

\begin{outline}
	\1 Legyen $R$ egy E.I.T. és legyen $R$-nek legalább $n+1$ eleme (pl. végtelen)
	\1 Ha $\deg f_1 \le n$ és $\deg f_2 \le n$ és $\forall r \in R: f_1(r) = f_2(r)$
	\1 Akkor $f_1 = f_2$ (azaz az együtthatók és tehát a fokszámok megegyeznek)
	\1 Bizonyítás: tegyük fel, hogy $f_1 \ne f_2$
		\2 $f_1 - f_2 \ne 0$ és $\deg f_1 - f_2 \le n$ $\implies$ max $n$ gyöke van (előző tétel)
		\2 Tehát $(f_1-f_2)(r)$ nem lehet nulla $n+1$ helyen, azaz ellentmondás
\end{outline}

\pagebreak

\subsection{Lagrange interpoláció}

\begin{outline}
	\1 Legyen $R$ egy test és $\deg f \le n$
	\1 Ha $n+1$ helyen ismerem $f(r)$ értékét: $y_i=f(x_i)$ \;\; ($i=1..n+1$)
	\1 Akkor $f$ egyértelműen megadható polinomok egyenlősége tétel miatt:\\
	$\exists! f \in R[x]: \forall i \in [1,n+1]: f(x_i) = y_i$
	\1 Bizonyítás:
		\2 Legyen $l_i(x) = (\; \prod_{j=0 \wedge i \ne j}^{n} (x-x_j) \;) \;/\;
		(\; \prod_{j=0 \wedge i \ne j}^{n} (x_i-x_j) \;)$
		\2 Ekkor $l_i(x)$ akkor $1$ ha $i=j$ egyébként mindig $0$
		\2 Így ez megoldás: $f(x) = \sum_{i=0}^{n} y_i*l_i(x)$
\end{outline}

\section{Többváltozós polinomok}

\begin{outline}
	\1 pl.: $4x^2+3xy+2y+1$
	\1 $R[x_1,x_2,...,x_n] = R[x_1][x_2]...[x_n]$ 
\end{outline}

\section{Egység (más fogalom, mint az egységelem)}

\begin{outline}
	\1 Egy együttható egység, ha $R$ minden elemének osztója.
		\2 Ekvivalens megfogalmazás: egység, ha létezik multiplikatív inverze
	\1 Egy polinom egység, ha minden polinomnak az osztója.
		\2 Test feletti polinomgyűrű: pontosan a nemnulla konstans polinomok.
	\1 Gyűrűelemet egységgel szorozva annak osztói, többszörösei nem változnak.
	\1 Egy EIT két elemét asszociáltaknak nevezzük, ha egymás egységszeresei.
		\2 Ez egy ekvivalencia (reflexív, szimmetrikus, tranzitív) reláció.
\end{outline}

\pagebreak

\section{Felbonthatatlan (irreducibilis) polinomok}

\begin{outline}
	\1 $f \in R[x]$ irreducibilis, ha:
		\2 $f \ne 0$ és $f$ nem egység
		\2 $f=g*h \implies$ $g$ egység vagy $h$ egység
	\1 Példa irreducibilis-ra $\mathbb{Q}$-ban: $(x^2+1)=\frac{1}{2}*(2x^2+2)$ \;\; ($\frac{1}{2}$ egység)
	\1 Példa nem irreducibilis-ra: $(x^2-1)=(x-1)(x+1)$
	\1 Test felett minden elsőfokú polinom felbonthatatlan.
\end{outline}

\subsection{Algebra alaptétele $\mathbb{C}$-ben}

\begin{outline}
	\1 Ha $f \in \mathbb{C}[x]$ és $\deg f \ge 1$ akkor létezik $f$-nek gyöke
	\1 Nem bizonyítjuk, nagyon nehéz
	\1 Következmény: $\mathbb{C}$-ben irreducibilis $\Leftrightarrow$ elsőfokú polinom
\end{outline}

\subsection{Irreducibilis $\mathbb{R}$-ben}

\begin{outline}
	\1 $\mathbb{R}$-ben azok és csak azok az $f$ polinomok irreducibilisak, amik:
		\2 $\deg f = 1$
		\2 $\deg f = 2$ és $f=ax^2+bx+c$ és $b^2-4ac < 0$
\end{outline}

\subsection{Irreducibilis $\mathbb{Z}_p$-ben, $\mathbb{Q}$-ban, $\mathbb{Z}$-ben}

\begin{outline}
	\1 Tétel bizonyítás nélkül
	\1 Minden $n \ge 1$-re létezik $n$-ed fokú irreducibilis polinom
\end{outline}

\subsection{Felbonthatóság és gyökök kapcsolata test felett}

\begin{outline}
	\1 van gyöke $\Leftrightarrow$ létezik első fokú faktora (osztója)
	\1 $\deg f \ge 2$ és van gyöke $\implies$ felbontható
	\1 $\deg f = 2 \lor \deg f = 3$: felbontható $\Leftrightarrow$ van gyöke
\end{outline}

\pagebreak

\section{Modulo polinom}

\begin{outline}
	\1 Létezik ilyen
	\1 Elvégzem a műveletet, elosztom a modulo-val és a maradékot veszem
	\1 Ha $f$ irreducibilis, akkor $mod \; f$ egy test
	\1 $\mathbb{C}$: mintha $mod \; i^2+1$-ben számolnánk
	\1 Euklideszi algoritmussal megoldható pl. $(3+2x)*g \equiv 1 \mod x^2+1$\\
	$(3+2x)^{-1} \equiv \frac{4}{13} * (-\frac{1}{2}x + \frac{3}{4}) \mod x^2+1$
\end{outline}

\subsection{Véges testek}

\begin{outline}
	\1 Alkalmazás: kódolás, hibajavítás
	\1 Véges test: egyszerre $mod\;p$ ($p$ prím) és $mod\;f$ ($\deg f = n$)
	\1 Ekkor $p^n$ elemű (elemszámú) testről beszélünk
	\1 pl.: $(2x+1)(x+2) \equiv 2x \;\; (mod \; 3, \; mod \; x^2+1)$\\
		$3^2=9$ elem: $\{0,1,2,x,x+1,x+2,2x,2x+1,2x+2\}$
\end{outline}

\section{Elem rendje, gyűrű karakterisztikája}

\subsection{Elem additív rendje}

\begin{outline}
	\1 Legyen $R$ egy gyűrű és $0 \ne r \in R$
	\1 $r$ rendje a legkisebb olyan $n$ egész, amelyre $n*r=0$
	\1 pl.: $Z_7$-ben $3$ rendje $7$ mert $7*3=21 \; mod \; 7 = 0$
	\1 Nullosztómentes gyűrűben az összes nemnulla elem rendje megegyezik.
\end{outline}

\subsection{Gyűrű karakterisztikája}

\begin{outline}
	\1 $char(R)$: nullosztómentes $R$ gyűrű karakterisztikája
		\2 $char(R)=0$ ha $R$ elemeinek közös rendje nem véges
		\2 egyébként $char(R)=$ $R$ elemeinek közös additív rendje
	\1 Példa: $char(\mathbb{Z}_p) = p$ ha $p$ prím
	\1 Példa: $char(\mathbb{Z})=char(\mathbb{Q})=char(\mathbb{R})=char(\mathbb{C})=0$
\end{outline}

\pagebreak

\section{Algebrai derivált}

\begin{outline}
	\1 Legyen $f \in R[x]$ ahol $R$ test és $f=a_0+a_1x+a_2x^2+...+a_nx^n$
	\1 Ekkor az algebrai derivált: $f'=a_1+2a_2x+3a_3x^2+...+n*a_nx^{n-1}$
		\2 $f'=\sum_{k=0}^{n} k*f_k*x^{k-1}$
\end{outline}

\section{Gyökök multiplicitása}

\begin{outline}
	\1 $c$ min. $k$-szoros gyök: $\exists q: f=(x-c)^k*q$
	\1 $c$ pontosan $k$-szoros gyök: $c$ $k$-szoros gyök, de nem $k+1$-szeres
\end{outline}

\subsection{Algebrai deriválttal összefüggés}

\begin{outline}
	\1 $f$-nek a $c$ $k$-szoros gyöke $\implies$ $f'$-nek $c$ min. $(k-1)$-szeres gyöke
	\1 Bizonyítás:
		\2 $f=(x-c)^k*q$
		\2 $f'=((x-c)^k*q)'=$nem biz$=((x-c)^k)'*q+(x-c)^k*q'=$\\
		$=k*(x-c)^{k-1}*q+(x-c)^k*q'=(x-c)^{k-1}*(k*q+(x-c)*q')$
	\1 Ez a tétel "min" helyett "pontosan"-nal akkor működik, ha $char(R) \not |\; k$
		\2 pl. $\mathbb{R}$ esetén, mivel $char(\mathbb{R})=0 \not |\; k$ (minden $k \ne 0$)
\end{outline}

\subsection{LNKO-val, algebrai deriválttal összefüggés}

\begin{outline}
	\1 $LNKO(f,f')=d=1 \implies$ $f$-nek nincs többszörös gyöke
	\1 $char(R)=0 \implies$ $d$ gyökei $f$ legalább kétszeres gyökei
\end{outline}

\subsection{Számolása Horner táblázattal}

\begin{outline}
	\1 Egy Horner táblázatot újrahasználhatunk
		\2 Mindig kevesebb oszlop kell, ezért mindig eggyel több $\times$-ot rakunk
		\2 Minden (pl. a maradék) azonos oszlopban van
	\1 Az előző eredmény mindig az új osztandó polinom
	\1 Addig írunk új sort, amíg a maradék $0$
\end{outline}

\pagebreak

\section{Euklideszi algoritmus polinomokkal}

\begin{outline}
	\1 Legyen $R$ egy test és $f,g \in R[x]$
	\1 Ha $d=LNKO(f,g)$ akkor
		\2 $d|f$ és $d|g$ \;\; ($d$ közös osztó)
		\2 $\forall h: (h|f) \wedge (h|g) \implies h|d$ \;\; ($d$ a legnagyobb)
	\1 A legnagyobb közös osztó egészek körében sem volt egyértelmű ($x$, $-x$)
	és polinomok esetén sem az.
	\1 Bővített euklideszi algoritmus is jó: $LNKO(f,g)=d=f*u+g*v$
\end{outline}

\subsection{Euklideszi algoritmus a gyakorlatban}

\begin{outline}
	\1 Egységgel be szabad szorozni, az nem változtat az eredményen.
	\1 Egymást követő osztások:
		\2 $f,g,r_1,r_2,...$ kettesével osztása megadja a következőt
		\2 1.: $f/g \implies f=g*q_1+r_1$
		\2 2.: $g/r_1 \implies g=r_1*q_2+r_2$
		\2 3.: $r_1/r_2 \implies r_1=r_2*q_3+r_3$
	\1 Addig osztunk, amíg a maradék 0 nem lesz.\\
	Ekkor az utolsó $\ne 0$ maradék a megoldás (LNKO).
		\2 pl.: $r_2 \ne 0 \wedge r_3 = 0 \implies LNKO(f,g)=d=r_2$
	\1 Bővített euklideszi algoritmus: pontosan úgy, mint skalárokkal
\end{outline}

\subsection{Kétváltozós diofantikus egyenletek}

\begin{outline}
	\1 Legyen $f*u+g*v=h$
	\1 Bővített euklideszivel ki kell számolni: $f*u'+g*v'=d=gcd(f,g)$
		\2 Itt is pontosan akkor oldható meg, ha $d|h$
	\1 Megoldások, ahol $w \in R[x]$ egy tetszőleges polinom:
		\2 $u_w = u_0+\frac{g}{d}w$ ahol $u_0=u'*\frac{h}{d}$
		\2 $v_w = v_0-\frac{f}{d}w$ ahol $v_0=v'*\frac{h}{d}$
\end{outline}

\pagebreak

\section{Hibakorlátozó kódolás}

\subsection{Alapfogalmak}

\begin{outline}
	\1 $\Sigma$ az ábécé, azaz egy rögzített véges halmaz
		\2 $n$ hosszú szavak halmaza: $\Sigma^n$
	\1 Hamming-távolság $2$ $n$ hosszú szó között: pozíciók száma, ahol különböznek
		\2 Szóhalmaz távolsága: min\{ bármely $2$ szavának távolsága \}
	\1 Kód: $\Sigma^n$ részhalmaza (azaz bizonyos $n$ hosszú szavak)
		\2 Kódszó: kód egy eleme
	\1 Szándék: küldeni kívánt szó (ezt alakítjuk kódszóvá)
	\1 Dekódolás: üzenethez legközelebbi kódszó kiválasztása
\end{outline}

\subsection{t-hibajelző, t-hibajavító kód}

\begin{outline}
	\1 t-hibajelző kód: max t helyen sérült üzenetnél észreveszi, hogy sérült
		\2 $d$ távolságú kód $\implies$ $d-1$ hibát jelez
		\2 Bizonyítás: távolság $d$ $\implies$ $d-1$ változás nem adhat új kódszót
	\1 t-hibajavító kód: max t helyen sérül üzenetnek tudja az eredeti kódszavát
		\2 $d$ távolságú kód $\implies$ $\lfloor \frac{d-1}{2} \rfloor$ hibát javít
		\2 Bizonyítás:
			\3 Legyen $w$ az eredeti kódszó, $w'$ a sérült, $w_2$ egy másik kódszó
			\3 Kiindulás: $distance(w,w') \le \lfloor \frac{d-1}{2} \rfloor$
			\3 Legyen $distance(w',w_2)=x$
			\3 Be kell látni: $\lfloor \frac{d-1}{2} \rfloor < x$
			\3 $distance(w,w_2) \ge d \implies x + \lfloor \frac{d-1}{2} \rfloor \ge d$
			\3 Átrendezve: $x \ge \lfloor \frac{d+1}{2} \rfloor > \lfloor \frac{d-1}{2} \rfloor$
	\1 Példa béna, pazarló kódra: n-szeres ismétlés $\implies$ kód távolsága n
\end{outline}

\pagebreak

\subsection{Singleton-korlát}

\subsubsection{Tétel}

\begin{outline}
	\1 Legyen
		\2 $Q=|\Sigma|$ az abécé mérete
		\2 $c$ a kódszavak száma
		\2 $n$ a kódszavak hossza
		\2 $d$ a kód távolsága
	\1 Állítás: $c \le Q^{n-d+1} = Q^{n-(d-1)}$
	\1 Bizonyítás
		\2 A kód távolsága $d$ $\implies$ $d-1$ változtatás nem adhat új kódszót
		\2 Változtassuk az összes kódban az utolsó $d-1$ betűt ugyan arra
		\2 Így is páronként különböznek, szóval el is hagyható a $d-1$ betű
		\2 Szóval max annyi kódszó van, hogy a kódszavakat $d-1$ betűvel rövidítve is páronként különbözik mindegyik
\end{outline}

\subsubsection{Következmény}

\begin{outline}
	\1 Hibajavító betűk száma $\ge d-1$
		\2 Azaz legalább ennyivel kell a szándéknál több betűt használni
	\1 MDS-kód: egyenlőség áll fenn a Singleton-korlát tételében
\end{outline}

\subsubsection{Reed-Solomon kód}

\begin{outline}
	\1 Egy MDS-kód
	\1 Működés
		\2 $\Sigma=\{f\in R\}$ ahol $R$ egy véges test
		\2 Szándék: $f$ polinom együtthatói
		\2 Kód: $f*g$
		\2 $g$ kódpolinom $d-1$ fokú
	\1 Tétel: $g$ gyökei páronként különböznek $\implies$ $d$ távolságú a kód
\end{outline}

\pagebreak

\subsection{Hamming-korlát}

\subsubsection{Tétel}

\begin{outline}
	\1 Legyen
		\2 $Q=|\Sigma|$ az abécé mérete
		\2 $c$ a kódszavak száma
		\2 $n$ a kódszavak hossza
		\2 $t$ hibát javít a kód ($\implies 2t+1$ vagy $2t+2$ távolságú)
	\1 Állítás: $c * \sum_{k=0}^{t}(\binom{n}{k} * (Q-1)^k) \le Q^n$
		\2 Szummán belül: kódszótól pontosan $k$ távolságra hány kódszó van
		\2 Válasz: $\binom{n}{k}=$"hol változtatok" $*$ "mire változtatok"$=(Q-1)^k$
		\2 Szóval a szumma: kódszótól $\le t$ távolságra lévő szavak száma
		\2 $Q^n$ pedig az ábécéből kirakható $n$ hosszú szavak száma.
	\1 Bizonyítás
		\2 A szumma eredményében "megszámolt" szavak diszjunktak különböző "kiindulási" kódszó esetén, hiszen $t$ hibát javít a kód.
		\2 Így a (diszjunkt halmazok száma) $*$ (diszjunkt halmaz mérete) nem lehet nagyobb, mint a lehetséges szavak száma.
\end{outline}

\subsubsection{Következmény}

\begin{outline}
	\1 Perfekt kód: egyenlőség áll fenn a Hamming-korlát tételében
\end{outline}

\pagebreak

\section{ZH 2 összefoglaló}

\subsection{Gyűrű, integritási tartomány, test}

\begin{outline}
	\1 Gyűrű: $*$ disztributív $+$-ra, $+$ asszoc. és kommutatív, $*$ asszoc. és $\exists 0$
	\1 Integritási tartomány: gyűrű, $*$ kommutatív, nullosztómentes (pl. $\mathbb{Z}_{prim}$)
	\1 E.I.T.: integritási tartomány, $\exists 1$
	\1 Test: E.I.T. ahol van osztás ($\forall a \ne 0: \exists a^{-1}$) (pl. $\mathbb{Q}, \mathbb{Z}_{prim}$)
	\1 Gyűrű elem (additív) rendje: legkisebb $n$ egész, hogy $n*$ az elem  $=0$
	\1 $char(R)$: I.T. elemeinek (megegyező) rendje (pl. $char(\mathbb{Z}_{prim})=prim$)
		\2 $char(R)=0$ ha az elemek rendje nem véges, pl. $char(\mathbb{Z})$, $char(\mathbb{Q})$
\end{outline}

\subsection{Polinom alapok}

\begin{outline}
	\1 $f = f_0 + ... + f_{\deg f}*x^{\deg f}$ ahol $\deg f=$ legnagyobb $n$ ahol $f_n \ne 0$
		\2 $f = 0 \implies \deg f = -\infty$
	\1 $\deg f+g \le max(\deg f, \deg g)$
	\1 $\deg f*g \le \deg f + \deg g$ (egyenlő, ha $R$ egy I.T, azaz nullosztómentes)
	\1 Maradékos osztás: $f=g*q+r \wedge \deg r < \deg q$: ($R$ egy E.I.T, $\exists {g_{\deg g}}^{-1}$)
		\2 Bizonyítások: egyértelműség: fokszámmal; létezés: rekurzióval
		\2 Gyakorlatban: rekurzióval, $f^*=f-g*f_{\deg f}*g_{\deg g}^{-1}*x^{\deg f - \deg g}$
			\3 Alapeset, megállás: $\deg f < \deg g$ (ekkor $r=f$)
			\3 $q=\sum f_{\deg f}*g_{\deg g}^{-1}*x^{\deg f - \deg g}$ (ahol $f$ persze változik)
	\1 Gyökök: $c$ gyök ha $f(c)=0$, ekkor $f=q*(x-c)$
		\2 Ha $R$ egy E.I.T. akkor $f$-nek legfeljebb $\deg f$ darab gyöke van
	\1 Egység: mindennek az osztója (polinom egység, ha $\forall$ polinomnak osztója)
		\2 $R$ egy test: pontosan nemnulla konstans polinomok az egységek
		\2 Gyűrűelem egységgel szorzása: osztói, többszörösei nem változnak.
		\2 Egy gyűrű elemei asszociáltak, ha egymás egységszeresei
	\1 Algebrai derivált: $f'=\sum_{k=0}^{\deg f} k * f_k * x^{k-1}$
\end{outline}

\pagebreak

\subsection{Horner-elrendezés (Horner táblázat)}

\begin{table}[h!]
	\centering
	\begin{tabular}{|c|c|c|c|c|c|c|}
		\hline
		& $f_n$ & $f_{n-1}$ & $f_{n-2}$ & ... & $f_0$ & \\
		\hline
		c & $\times$ & $c_1=f_n$ & $c_2=c_1*c+f_{n-1}$ & .. & $c_n=c_{n-1}*c+f_1$ & $c_{n+1}=...=f(c)$ \\
		\hline
	\end{tabular}
\end{table}

\subsection{Felbonthatatlan (irreducibilis) polinomok}

\begin{outline}
	\1 $f \ne 0$ irreducibilis ha nem egység és $f=g*h \implies$ $g$ vagy $h$ egység
	\1 Test felett: $\deg f = 1 \implies f$ irreducibilis
	\1 $\mathbb{C}$-ben: irreducibilis $\Leftrightarrow$ elsőfokú polinom (algebra alaptétele $\mathbb{C}$-ben)
	\1 $\mathbb{Z}_{prim}$, $\mathbb{Q}$, $\mathbb{Z}$: minden $n \ge 1$-re létezik $n$-ed fokú irreducibilis polinom
\end{outline}

\subsection{Gyökök multiplicitása}

\begin{outline}
	\1 $c$ legalább $k$-szoros gyök $\implies \exists q: f=(x-c)^k*q$
	\1 $f$-nek a $c$ $k$-szoros gyöke $\implies$ $f'$-nek $c$ min. $(k-1)$-szeres gyöke
		\2 Bizonyítás: $f=(x-c)^k*q$ deriválása (szorzat deriváltja azonosság)
		\2 "min" helyett "pontosan" ha $char(R) \not | \; k$ (mert pl. $char(\mathbb{R})=0$)
	\1 $char(R)=0 \implies LNKO(f,f')$ gyökei az $f$ legalább kétszeres gyökei
\end{outline}

\subsection{(Bővített) euklideszi algoritmus}

\begin{outline}
	\1 Test felett működik; nem egyértelmű: egységgel részeredményeket megszorozhatjuk
	\1 Gyakorlatban: $f,g,r_1,r_2,...$ kettesével osztása megadja a következőt
	\1 Ugyan úgy megy, mint skalárokkal; utolsó nemnulla az eredmény
\end{outline}

\subsection{Kétváltozós diofantikus egyenletek}

\begin{outline}
	\1 $u,v=?$ \;\; és \;\; $f*u+g*v=h$ \;\; és \;\; $f*u'+g*v'=d=gcd(f,g)$
	\1 Megoldható $\Leftrightarrow d|h$
	\1 Megoldások, ahol $w \in R[x]$ tetszőleges:
		\2 $u_w = u_0+\frac{g}{d}w$ ahol $u_0=u'*\frac{h}{d}$
		\2 $v_w = v_0-\frac{f}{d}w$ ahol $v_0=v'*\frac{h}{d}$
\end{outline}

\end{document}
