% !TeX spellcheck = hu_HU
\documentclass[12pt,a4paper]{article}
\usepackage[utf8]{inputenc}
\usepackage{cmap}
\usepackage[T1]{fontenc}
\usepackage[magyar]{babel}
\usepackage{amsmath}
\usepackage{amsfonts}
\usepackage{amssymb}
\usepackage{graphicx}

\usepackage{struktex}
\usepackage{outlines}
\usepackage{hyperref}

\hyphenpenalty=10000

\begin{document}

\begin{center}
	\huge
	Programozáselmélet\\
	\vspace{1mm}
	\LARGE
	Előadás jegyzet\\
	\vspace{5mm}
	\large
	Készült Borsi Zsolt előadásai\\
	és Gregorics Tibor gyakorlatai alapján\\
	\vspace{5mm}
	Sárközi Gergő, 2021-22-1. félév\\
	Nincsen lektorálva!
\end{center}

\tableofcontents

\pagebreak

\section{Bevezetés}

\paragraph{Intervallum} $ [a..b] ::= \{ x \in \mathbb{Z} | a \le x \wedge x \le b \}$ \\
Üres, ha $a > b$.

\paragraph{Reláció, rendezett pár} $ A \times B ::= \{ (a,b) | a \in A \wedge b \in B \} $ \\
Az $A \times B$ halmaz minden részhalmazát, az üres halmazt is, relációnak nevezzük. \\
$ | A \times B | = |A| * |B| $ \\
A rendezett pár felfogható relációként is: $A$-ról $B$-re képezés. \\
Jelölése: $R \subseteq A \times B$ \\
Legyen $A={1,2,3}$, $B={a,b,c,d}$ és $R={(1,a), (1,b), (2,c)}$. \\
Ekkor $R \subseteq A \times B$, valamint $D_R = \{1,2\}$, és $R(1)=\{a,b\}$. \\
$R(x) = \emptyset \Leftrightarrow |R(x)|=0 \Leftrightarrow x \notin D_R $ \\
$D_R ::= \{ a \in A | \exists b \in B: (a,b) \in R \}$ \\
$R_R ::= \{ b \in B | \exists a \in A: (a,b) \in R \}$ \\
$R(a) ::= \{ b \in B | (a,b) \in R \} $ \\
Reláció inverze: \\
$R^{-1} = \{(b,a) \in B \times A | (a,b) \in R\}$

\paragraph{Függvény, determinisztikus reláció} -\\
Legyen $A$ és $B$ nemüres halmaz.
$R \subseteq A \times B$ függvény, ha $\forall a \in A: |R(a)| \le 1$. \\
Függvény jelölése: $R \in A \to B$ (ekkor $|R(a)| \le 1$) \\
vagy $R: A \to B$ (ekkor $|R(a)| = 1$, azaz $D_R=A$) \\
$R: A \to B$ esetén sokszor $R(a)=\{b\}$ helyett csak $R(a)=b$-t használunk

\paragraph{Kompozíció} $P \subseteq A \times B$ és $Q \subseteq B \times C$\\
Normál: \\
$Q \circ P=\{(a,c) \in A \times C \;|\; \exists b \in B: (a,b) \in P \wedge (b,c) \in Q\}$\\
Szigorú: ($\exists (a,b) \in P: \exists c \in C: (b,c) \not \in Q \implies (Q \odot P)(a) = \emptyset$)\\
$Q \odot P=\{(a,c) \in A \times C \;|\; \exists b \in B: (a,b) \in P \wedge (b,c) \in Q \wedge P(a) \subseteq D_Q\}$

\paragraph{Sorozatok} $H^{**} = H^* \cup H^\infty $ \\
$H = \{1,2,3\}$ \\
$<1,5,3> \notin H^*$, mert $5 \notin H$ \\
$<1,1,3> \in H^*$ \\
$<1,1,1,1,...> \notin H^*$, viszont $\in H^\infty$ \\

\paragraph{Egyéb} -\\
hamis mindent implikál: (HAMIS $\implies$ bármi) = IGAZ

\pagebreak

\section{Alapfogalmak}

\subsection{Állapottér}

\begin{outline}
	\1 Állapot: változókkal címkézett értékek halmaza
		\2 példa: $a=\{ v_1:a_1, v_2:a_2 \}$
		\2 változó használata: $n(\{n:6\})=6$ (valójában $\{6\}$, de elnézzük)
	\1 Érték: típusérték-halmaz egy eleme
	\1 Állapottér: összes lehetséges állapot halmaza
		\2 jelölés: $A=( v_1:A_1, v_2:A_2 )$
	\1 Az $A$ állapottérnek altere a $B$ állapottér ($B<A$),
	ha a $B$ állapottér típusértékei (számmosságot is figyelmbe véve) benne vannak $A$-ban is.
		\2 példa: $A=(x:\mathbb{N},y:\mathbb{N},z:\mathbb{N})$
		és $B=(y:\mathbb{N},x:\mathbb{N})$ akkor $B<A$.
		\2 A típusérték-halmazoknak meg kell egyezniük: az nem "elég",
		ha az egyik a másiknak csak a  részhalmaza. \\
		Itt egyik sem részhalmaza a másiknak: $A=(x:\mathbb{N)}$, $B=(x:\mathbb{Z})$
		\2 TODO lehet különböző a változó neve?
\end{outline}

\subsection{Feladat}

\begin{outline}
	\1 Feladat: reláció, amely egy állapottérről ugyan arra az állapottérre képez
		\2 Ha $A$ állapottér, akkor $F \subseteq A \times A$ egy feladat.
		\2 Tehát $F=\{\}$ is egy feladat.
\end{outline}

\subsubsection{Példa}

Adjuk meg egy pozitív egész szám egy pozitív osztóját. \\
$A=( n:\mathbb{N}^+, d:\mathbb{N}^+ )$ \\
$D_F = A$ \\
$F \subseteq A \times A$ \\
$F(\{n:6,d:11\}) = \{ \{n:6,d:2\}, \{n:6,d:3\}, ... \} $ \\
Itt a kimenetnél nézőpont kérdése, hogy n=6 feltétel-e, vagy meg lehetne változtatni. \\
$F(x,y)=\{(u,v) ; u=x \wedge v|x\}$ \\
$F(x,*)=\{(*,v) ; v|x\}$ \\
$F = \{(u,v) \in A \times A \;|\; d(v)|n(u) \}$

\pagebreak

\subsection{Program}

\begin{outline}
	\1 Leképezés, amely megadja bármelyik kiinduló állapotra az abból induló végrehajtási sorozatokat.
	\1 Program: $S \subseteq A \times (\overline{A} \cup \{fail\})^{**}$ ahol:
		\2 $\overline{A} = \underset{A \le B}{\bigcup} B $
		\2 $D_S=A$ (minden bemenetet lekezel)
		\2 $\forall a \in A: \forall vs \in S(a): |vs| \ge 1 \wedge vs_1 = a$
		(végrehajtási sorozat nem üres és első tagja a bemenet)
		\2 $\forall vs \in R_S: (\forall i \in \mathbb{N}^+: i < |vs| \to vs_i \neq fail)$
		(csak az utolsó állapot lehet fail)
		\2 $\forall vs \in R_S: |vs|<\infty \to vs_{|vs|} \in A \cup \{fail\}$
		(az utolsó állapot, ha van ilyen, csak alap-állapottérbeli vagy fail lehet)
\end{outline}

\subsubsection{Alap-állapottér}

\begin{outline}
	\1 nincsenek benne a segédváltozók
	\1 kiindulási állapot mindig alap-állapottérbeli
	\1 végállapot alap-állapottérbeli, vagy fail
\end{outline}

\subsubsection{Végrehajtási sorozat}

\begin{outline}
	\1 első tagja mindig a bemeneti állapot
	\1 ha van benne fail, akkor csak leghátul lehet
	\1 segédváltozók a végrehajtási sorozat első és utolsó tagjában nincsenek, viszont a többiben igen
	\1 legalább 1, max. végtelen hosszú
	\1 egy kezdő állapotból több különböző végrehajtási sorozat indulhat
	\1 értékadás, ami nem változtat semmit is generál új végrehajtási sorozatot
\end{outline}

\subsubsection{Példa}

$A=(x:\{1,2,3\})$ \\
S program (azaz halmaz) elemei: (mindegyik felírás helyes)\\
$\{x:1\} \to <\{x:1\}, \{x:3\}>$ \\
$1 \to <1,1,1>$ \\
$(2) \to <(2), (3), (4)>$ \\
$(3, <3, fail>)$

\subsection{Programfüggvény}

\begin{outline}
	\1 megadja egy program lehetséges végállapotait (amik sem fail, sem végtelenek nem lehetnek) egy adott kezdőállapot esetén
	\1 Programfüggvény: egy reláció, pl. $p(S) \subseteq A \times A$
		\2 ez a $S \subseteq A \times (\overline{A} \cup \{fail\})^{**}$ programhoz van
		\2 $D_{p(S)} = \{ a \in A | S(a) \subseteq \overline{A}^* \}$
		(ott értelmes, ahol S csak véges, csak nem fail végrehajtási sorozatot ad)
		\2 $\forall a \in D_{p(S)}: p(S)(a) = \{b \in A | \exists vs \in S(a): b=vs_{|vs|}\}$
		(azaz $p(S)(x)$ = 'S program lehetséges végállapotjai (halmaza) $x$ kezdőállapot esetén')
		\2 ha $x \notin D_{p(S)}$, akkor $p(S)(x) = \emptyset$
\end{outline}

\subsection{Megoldás}

\begin{outline}
	\1 S program megoldja az F feladatot (S program teljesen helyes F feladatra nézve) ha:
		\2 $D_F \subseteq D_{p(S)}$ (feladat minden kezdőállapotát le tudja kezelni)
		\2 $\forall a \in D_F : p(S)(a) \subseteq F(a)$
		(program végállapotja, bármilyen feladatban értelmezett kezdőállapot esetén,
		részhalmaza a feladat "megoldáshalmazának")
\end{outline}

\pagebreak

\subsection{Parciális helyesség}

\begin{outline}
	\1 Nem hasznos, nem nagyon fogjuk használni.
	\1 Gyenge programfüggvény:
		\2 $\widetilde{p}(S) \subseteq A \times (A \cup \{fail\})$ reláció
		\2 $S \subseteq A \times (\overline{A} \cup \{fail\})^{**}$ programhoz való
		\2 $D_{\widetilde{p}(S)} = \{ a \in A | S(a) \cap (\overline{A} \cup \{fail\})^* \ne \emptyset \}$
		(ott értelmes, ahol S ad véges végállapotot is)
			\3 fail-t megenged, de végtelent nem
			\3 értelmezve van, ha azonos bemenethez van végtelen és véges megoldás is
		\2 $\forall a \in D_{\widetilde{p}(S)}: \widetilde{p}(S)(a) =$\\
		$\{b \in A \cup \{fail\} | \exists vs \in S(a) \cap (\overline{A} \cup \{fail\})^*: b=vs_{|vs|}\}$ \\
		(azaz $\widetilde{p}(S)(x)$ = 'S program lehetséges végállapotjai (halmaza) $x$ kezdőállapot esetén, csak véges vs-eket nézve')
	\1 Parciális helyesség ($S$ program parciálisan helyes az $F$ feladatra nézve):
		\2 ha $\forall a \in D_F: \widetilde{p}(S)(a) \subseteq F(a)$
		\2 szóval végtelen vs-ek kivételével: mint a rendes megoldás
\end{outline}

\subsection{Elemi programok}

\begin{outline}
	\1 Elemi program definíció: $\forall a \in A: S(a) \subseteq$ \\
	$\{ <a>, <a,fail>, <a,a,a,...>, <a,b> | b \in A \}$
	\1 Nevezetes elemi programok:
		\2 $\forall a \in A: ABORT(a) = \{<a, fail>\}$
		\2 $\forall a \in A: SKIP(a) = \{<a>\}$
		\2 értékadás (x := 1)
		\2 szimultán értékadás (x,y := y,x)
		\2 értékkiválasztás (x :$\in$ \{n-1, n-2\})
\end{outline}

\pagebreak

\subsection{Programok, programfüggvények, stb. relációi}

\subsubsection{$S_1 \subseteq S_2 \implies D_{p(S_2)} \subseteq D_{p(S_1)}$}

Bizonyítás: \\
$a \in D_{p(S_i)} \Leftrightarrow S_i(a) \subseteq \overline{A}^*$ (def) \\
$S_1 \subseteq S_2 \implies S_1(a) \subseteq S_2(a)$ \\
$(S_2(a) \subseteq \overline{A}^*) \wedge (S_1(a) \subseteq S_2(a))
\implies (S_1(a) \subseteq (S_2(a) \subseteq \overline{A}^*)$

\subsubsection{$S_1 \subseteq S_2 \implies \forall a \in D_{p(S_2)}: p(S_1)(a) \subseteq p(S_2)(a)$}

Bizonyítás: \\
$x \in p(S_i)(a) \Leftrightarrow \exists vs \in S_i(a): vs_{|vs|}=x$ (def) \\
$S_1 \subseteq S_2 \implies S_1(a) \subseteq S_2(a)$ \\
$(\exists vs \in S_1(a): vs_{|vs|}=x) \wedge (S_1(a) \subseteq S_2(a))
\implies (\exists vs \in S_2(a): vs_{|vs|}=x)$

\subsubsection{$S' \subseteq S$ és S megoldja F-et, akkor S' is}

Lásd előző kettő bizonyítás: \\
$D_F \subseteq D_{p(S)} \subseteq D_{p(S')}$ \\
$p(S')(a) \subseteq p(S)(a) \subseteq F(a)$

\subsubsection{$F \subseteq F'$ és S megoldja F'-t, attól még S nem oldja meg F-t}

Példa: F' nem determinisztikus, F kevesebb megoldást fogad el.

\subsubsection{Determinisztikusság relációk}

F, S és p(S) determinisztikussága nagyrészt (vagy egyáltalán?) nem következtethető ki egymásból.

\subsubsection{Szigorúbb feladat}

\begin{outline}
	\1 $F_2 \subseteq A \times A$ feladat szigorúbb, mint $F_1 \subseteq A \times A$, ha:
		\2 $D_{F_1} \subseteq D_{F_2}$ \;\;(több helyen értelmezett)
		\2 $\forall a \in D_{F_1}: F_2(a) \subseteq F_1(a)$ \;\;(kevesebb jó kimenet)
		\2 Tehát egy feladat önmagának is a szigorítása
	\1 Állítás: ha $S$ program megoldja $F_2$-t, akkor $F_1$-et is
	\1 Bizonyítás: $D_{F_1} \subseteq D_{F_2} \subseteq D_{p(S)}$\\
	és $\forall a \in D_{F_1}: p(S)(a) \subseteq F_2(a) \subseteq F_1(a)$
\end{outline}

\pagebreak

\section{Logikai függvények}

\begin{outline}
	\1 Legyen $R \in A \to \mathbb{L}$,
	ekkor $\lceil R \rceil = \{ a \in A | R(a) = {igaz} \}$
	\1 Igazsághalmaz: $\lceil R \rceil$
	\1 $D_R$ nem feltétlenül $A$ ($\in$ jelölés miatt), tehát attól még, hogy nem igaz valahol, nem feltétlenül hamis ott. Éppen ezért egy ilyen logikai függvényt nem lehet megadni csak az igazságtáblájával.
		\2 Viszont megegyezés miatt igazságtáblával megadott logikai függvény mindenhol értelmezve van.
	\1 $R : A \to \mathbb{L}$ viszont már a teljes $A$-n értelmezve van
	\1 Nevezetes logika függvények:
		\2 $\forall a \in A: IGAZ(a) = \{igaz\}$ \\
		$\lceil IGAZ \rceil = A$
		\2 $\forall a \in A: HAMIS(a) = \{hamis\}$ \\
		$\lceil HAMIS \rceil = \emptyset$
	\1 Példa: $A=(x:\mathbb{N})$, $R=(1<x<3)$
\end{outline}

\subsection{Műveletek logika függvényekkel}

\begin{outline}
	\1 Legyen $A$ felett $Q,R$ logikai függvények
	\1 $Q \wedge R = \{ a \in A | Q(a) \wedge R(a) \}$
	\1 $Q \lor R = \{ a \in A | Q(a) \lor R(a) \}$
\end{outline}

\subsection{Következik reláció, maga után vonás}

\begin{outline}
	\1 $Q,R \in A \to \mathbb{L}$ és $A \ne \emptyset$
	\1 $(Q \implies R) \Leftrightarrow (\lceil Q \rceil \subseteq \lceil R \rceil)$
		\2 ha Q(x) igaz, akkor R(x) is
	\1 Vegyünk tetszőleges $P \in A \to \mathbb{L}$ függvényt
		\2 $P \implies IGAZ$
		\2 $HAMIS \implies P$
\end{outline}

\pagebreak

\section{Leggyengébb előfeltétel}

\begin{outline}
	\1 Logikai függvény, amely pontosan ott igaz, ahonnan kiindulva $S$ hibátlanul terminál
	és az összes lehetséges végállapotban $R$ igaz.
	\1 Legyen $S \subseteq A \times (\overline{A} \cup \{fail\})^{**}$ program
	és $R \in A \to \mathbb{L}$ logika függvény
	\1 $lf(S,R) : A \to \mathbb{L}$
		\2 tehát minden A-beli ponthoz (pontosan egy) logikai értéket rendel
	\1 $\lceil lf(S,R) \rceil = \{ a \in A | a \in D_{p(S)}
	\wedge p(S)(a) \subseteq \lceil R \rceil \}$
	\1 Leggyengébb, mert $P \implies lf(S,R)$ esetén $P$ szigorúbb, mint $lf(S,R)$
\end{outline}

\subsection{lf tulajdonságai}

\begin{outline}
	\1 Legyen $A$ felett $S$ egy program és $Q,R$ logikai függvények
	\1 Általános tulajdonságok:
		\2 $lf(S,HAMIS) = HAMIS$
		\2 ha $Q \implies R$, akkor $lf(S,Q) \implies lf(S,R)$
		\2 $lf(S,Q) \wedge lf(S,R) = lf(S,Q \wedge R)$
		\2 $lf(S,Q) \lor lf(S,R) \implies lf(S,Q \lor R)$
		\2 $\lceil lf(S,IGAZ) \rceil = D_{p(S)}$
	\1 Elemi függvényekkel:
		\2 $\lceil lf(ABORT,R) \rceil = \emptyset$
		\2 $lf(ABORT,R) = HAMIS$
		\2 $\lceil lf(SKIP,R) \rceil = \lceil R \rceil$
		\2 $lf(SKIP,R) = R$
\end{outline}

\subsection{lf kiszámolása}

\begin{outline}
	\1 $\lceil lf(S,R) \rceil = \{a \in A \;|\; a \in D_{p(S)}$ és behelyettesítjük R-be S-t $\}$
	\1 pl.: $\lceil lf((x:=x+1),(x<10)) \rceil=\{a \in A \;|\; igaz \wedge (x<10)^{\leftarrow x:=x+1}\}=\{a \in A \;|\; x'=x(a)+1 \wedge x' < 10\}=\{a \in A \;|\; x(a)+1 < 10\} = (x<9)$
\end{outline}

\pagebreak

\section{Paramétertér, specifikáció tétele}

\subsection{Paramétertér}

\begin{outline}
	\1 $B$ az $F \subseteq A \times A$ feladat paramétertere, ha
		\2 létezik $F_1 \subseteq A \times B$ és $F_2 \subseteq B \times A$
		\2 hogy $F = F_2 \circ F_1$
	\1 Minden feladatnak létezik paramétertere:\\
	legyen $B=A$, $F_1 = id$ és $F_2 = F$
	\1 Minden feladatnak végtelen sok paramétertere van
	\1 paraméter: paramétertér egy eleme
\end{outline}

\subsection{Specifikáció tétele}

\begin{outline}
	\1 Legyen:
		\2 $S \subseteq A \times (\overline{A} \cup \{fail\})^{**}$ egy program
		\2 $F \subseteq A \times A$ egy feladat
		\2 $B$ az $F$ egy paramétertere
			\3 $F_1 \subseteq A \times B$ és $F_2 \subseteq B \times A$
			\3 $F = F_2 \circ F_1$
	\1 Definiáljunk kettő $A \to \mathbb{L}$ logikai fv-t minden paraméterhez ($\forall b \in B$)
		\2 $\lceil Q_b \rceil = F_1^{-1}(b)$ (olyan $a$-k, amikhez $F_1$ rendel egy $b$-t)
		\2 $\lceil R_b \rceil = F_2(b)$ (olyan $a$-k, amiket $F_2$ rendel egy $b$-hez)
	\1 Ekkor: $(\forall b \in B: Q_b \implies lf(S,R_b)) \implies$ ($S$ megoldja $F$-et)
	\1 Szóval ez egy elégséges, de nem szükséges feltétel
	\1 Előfeltétel gyengítésével és/vagy utófeltétel szigorításával egy \textit{szigorúbb feladat}ot kapunk.
\end{outline}

\pagebreak

\subsubsection{Bizonyítás: $(\forall b \in B: Q_b \implies lf(S,R_b)) \implies$ ($S$ mo. $F$-et)}

\begin{outline}
	\1 $D_F \subseteq D_{p(S)}$
		\2 $a \in D_F \implies \exists b \in B: (a,b) \in F_1 \implies a \in \lceil Q_b \rceil$
		\2 $\lceil lf(S,R_b) \rceil = \{a \in D_{p(S)} | p(S)(a) \subseteq \lceil R_b \rceil\} \subseteq D_{p(S)}$
		\2 $(Q_b \implies lf(S,R_b)) \implies \lceil Q_b \rceil \subseteq \lceil lf(S,R_b) \rceil \subseteq D_{p(S)}$
		\2 beláttuk, hogy $a \in D_F \implies a \in \lceil Q_b \rceil \subseteq D_{p(S)}$
	\1 $\forall a \in D_F: p(S)(a) \subseteq F(a)$
		\2 $\lceil lf(S,R_b) \rceil = \{a \in D_{p(S)} | p(S)(a) \subseteq \lceil R_b \rceil\}$
			\3 tehát $p(S)(a) \subseteq \lceil R_b \rceil$
		\2 $F_2(b) \subseteq F(a)$
		\2 beláttuk, hogy $p(S)(a) \subseteq \lceil R_b \rceil = F_2(b) \subseteq F(a)$
\end{outline}

\subsection{Feladat specifikációja}

$F(n,*) = \{(n,d) \;|\; d|n\}$\\
$A=(n:\mathbb{N}^+, d:\mathbb{N}^+)$ (állapottér)\\
$B=(n':\mathbb{N}^+)$ (paramétertér)\\
$\forall b \in B: Q_b(a) = (n(a) = n'(b))$\\
$\forall b \in B: R_b(a) = (n(a) = n'(b) \wedge d(a)|n(a))$\\
Rövidebben:\\
$Q=(n=n')$ (előfeltétel)\\
$R=(Q \wedge d|n)$ (utófeltétel)

\subsection{Példa}

Adott egy egész szám. Növeljük meg 1-gyel!\\
$A=(x:\mathbb{Z})$\\
$B=(x':\mathbb{Z})$\\
$Q=(x=x')$\\
$R=(x=x'+1)$\\
Helyesség belátása:\\
$S=(x:=x+1)$, tehát ezt kell belátni: $Q \implies lf(x:=x+1, R)$\\
R-ben helyettesítés: $Q \implies (x+1=x'+1)$\\
Azaz: $(x=x') \implies (x+1=x'+1)$

\pagebreak

\section{Új jelölések}

\begin{outline}
	\1 Ha egy $l: \mathbb{L}$ igaz, akkor \textit{valami}: $l \to valami$
	\1 Intervallum: $\forall i \in [1..n]: ...$
	\1 Függvény: $f : \mathbb{Z} \to \mathbb{Z}$
	\1 Tömb: $x:\mathbb{Z}^n$, ahol $n \in \mathbb{N}$ (n-t nem kell külön bevezetni)
		\2 1-től indexelünk, meghozzá így: $x[1]$
\end{outline}

\section{Programkonstrukciók}

\subsection{Szekvencia}

\begin{outline}
	\1 Legyen $S_1$ és $S_2$ program $A$ állapottéren
	\1 $(S_1;S_2)$ reláció a két program szekvenciája
	\1 $(S_1;S_2)(a) =$\\
	$\{vs \in \overline{A}^\infty \;|\; vs \in S_1(a)\}$ \;\;\;($S_1$ végtelen vs-t ad)\\
	$\cup \{vs \in (\overline{A} \cup \{fail\})^* \;|\; vs \in S_1(a) \wedge vs_{|vs|}=fail\}$ \;\;\;($S_1$ vs vége fail)\\
	$\cup \{vs \in (\overline{A} \cup \{fail\})^{**} \;|\; vs=x \oplus y \wedge x \in S_1(a) \wedge |x| < \infty \wedge x_{|x|} \ne fail \wedge y \in S_2(x_{|x|}) \}$
	\;\;\;($S_1$ vs véges és nem fail, ekkor $S_2$-vel folytatjuk)
	\1 Nem adom meg a szekvencia állapotterét, csak a két programot: több lehetséges alap állapottér van, mert lehetnek segédváltozók.
	\1 TODO ezeket ellenőrizni (hivatalos jegyzetben nincsenek benne)
		\2 $p((S_1;S_2))=p(S_2) \odot p(S_1)$
		\2 $D_{p((S_1;S_2))}=\{a \in D_{p(S_1)} \;|\; p(S_1)(a) \subseteq D_{p(S_2)} \}$
\end{outline}

\subsubsection{Szekvencia mint megoldó program}

\begin{outline}
	\1 $S_1$ megoldja $F_1$-t, $S_2$ megoldja $F_2$-t, $S$ legyen $(S_1;S_2)$
	\1 $S$ nem oldja meg $F_2 \circ F_1$-t ($F_1$ legyen tágabb, stb... bonyolult)
	\1 $S$ megoldja $F_2 \odot F_1$-t (hosszú a bizonyítás, gyakon nem is volt MÉG)
\end{outline}

\pagebreak

\subsection{Elágazás}

\begin{outline}
	\1 Legyen $S_1,S_2,...,S_n$ programok, $\pi_1,\pi_2,...,\pi_n$ logikai függvények A felett
		\2 azaz $\pi_i \in A \to L$, nem pedig $\pi_i : A \to L$
	\1 $IF=IF(\pi_1:S_1,...,\pi_n:S_n)$
		\2 $\forall a \in A: IF(a) = w_0(a) \cup \bigcup_{i=1..n}w_i(a)$
		\2 $w_0(a)=\begin{cases}
		\{<a,fail>\} \text{\;ha\;} \forall i \in [i..n]: (a \in D_{\pi_i} \wedge \lnot \pi_i(a))\\
		\emptyset \text{\;különben}
		\end{cases}$
		\2 $w_i(a) = \begin{cases}
		S_i(a) \text{\;ha\;} a \in D_{\pi_i} \wedge \pi_i(a)\\
		\emptyset \;\;\;\;\; \text{\;ha\;} a \in D_{\pi_i} \wedge \lnot \pi_i(a)\\
		\{<a,fail>\} \text{\;ha\;} a \not\in D_{\pi_i}\\
		\end{cases}$
	\1 Megjegyzések
		\2 több $\pi_i$ egyszerre teljesül: mindet belerakjuk a halmazba
		\2 $\pi_i$ valahova nem rendel semmit: fail-t is rendelünk oda (de a vs első eleme a valami)
		(Emlékeztető: igazsághalmazzal megadott logikai fv megegyezés szerint mindenhova rendel valamit.)
		\2 összes $\pi_i$ egyszerre hamis: fail-t rendelünk oda (de a vs első eleme a valami)
\end{outline}

\subsubsection{Programfüggvény}

\begin{outline}
	\1 $D_{p(IF)}=\{a\in A \;|\; a \in \bigcap D_{\pi_i} \wedge a \in \bigcup\lceil \pi_i \rceil \wedge (a \in \lceil \pi_i \rceil \implies a \in D_{p(S_i)} )\}$
	\1 $p(IF)(a)=\bigcup_{i=1..n}\{ x \in p(S_i)(a) \; | \; a \in \lceil \pi_i \rceil\ \}$
\end{outline}

\pagebreak

\subsection{Ciklus}

\begin{outline}
	\1 Legyen $S$ program, $\pi$ logikai függvény $A$ felett
	\1 $DO(\pi:S)$ az elöl tesztelő ciklus ($S$ ciklusmag, $\pi$ ciklusfeltétel)
		\2 $DO(a)=\begin{cases}
			(S;DO)(a) \text{\;ha\;} a \in D_\pi \wedge \pi(a)\\
			<a> \;\;\;\;\;\;\; \text{\;ha\;} a \in D_\pi \wedge \lnot \pi(a)\\
			<a,fail> \text{\;ha\;} a \not \in D_\pi
			\end{cases}$
		\2 végtelen ciklus jelölése:
			\3 $DO(\pi:S)=\{1 \to <1,1,1...>\}$
			\3 $DO(\pi:S)=\{1 \to <1,(2,3,1)^{\infty}...>\}$
			\3 $DO(\pi:S)=\{1 \to <1,(2,3,1)^*,4>\}$ (*: véges sokszor)
	\1 Programfüggvények:
		\2 TODO $p(DO)$ és $D_{p(DO)}$ =?
		\2 $D_{p(S)} \not \subseteq D_{p(DO)}$ (mert: IGAZ és SKIP)
		\2 $D_{p(DO)} \not \subseteq D_{p(S)}$ (mert: HAMIS és ABORT)
\end{outline}

\pagebreak

\section{Levezetési szabályok}

\subsection{Szekvencia levezetési szabálya}

\begin{outline}
	\1 Legyen $S=(S_1;S_2)$ program és $Q,Q',R$ logikai függvény
	\1 Ha $Q \implies lf(S_1,Q')$ és $Q' \implies lf(S_2,R)$
	\1 Akkor (és csak akkor) $Q \implies lf(S,R)$
	\1 Magyarázat: $Q \implies lf(S,R)$ jelentése: $S$ elvisz $Q$-ból $R$-be
		\2 Szóval: $Q \to (S_1) \to Q' \to (S_2) \to R$
\end{outline}

\subsection{Elágazás levezetési szabálya}

\begin{outline}
	\1 Legyen $IF=IF(\pi_1:S_1,...,\pi_n:S_n)$ és $Q,R$ logikai függvény 
	\1 Ha
		\2 $Q \implies AND_{i=1}^{n} \pi_i \lor \lnot \pi_i$ \;\; (mindegyik mindenhol értelmes)
		\2 $Q \implies OR_{i=1}^{n} \pi_i$ \;\; (mindenhol legalább egy teljesül)
		\2 $\forall i \in [1..n]: Q \wedge \pi_i \implies lf(S_i,R)$ \;\; (bármelyik ágon $R$ igaz lesz)
	\1 Akkor $Q \implies lf(IF,R)$
\end{outline}

\subsection{Ciklus levezetési szabálya}

\begin{outline}
	\1 Legyen $DO=DO(\pi:S)$ és $P,Q,R$ logikai függvény, ahol $P$ invariáns\\
	és $t:A \to \mathbb{Z}$ termináló állítás
	\1 Ha
		\2 $Q \implies P$ \;\; (invariáns alapból teljesül)
		\2 $P \wedge \lnot \pi \implies R$ \;\; (jó helyen áll meg)
		\2 $P \implies \pi \lor \lnot \pi$ \;\; (ciklusfeltétel nem abortál)
		\2 $P \wedge \pi \implies t > 0$ \;\; (cikluson belül van még lépés)
		\2 $P \wedge \pi \wedge t = t_0 \implies lf(S,P \wedge t<t_0)$ \;\; (invariáns marad, lépés\;-\;-)
	\1 Akkor $Q \implies lf(DO,R)$
	\1 $P$ az invariáns állítás, $t$ a terminálófüggvény (hány lépés van hátra)
\end{outline}

\pagebreak

\section{Helyességbizonyítás}

\subsection{Módszere}

\begin{outline}
	\1 Be akarjuk látni, hogy $S$ megoldja az $A,B,Q,R$ specifikációjú feladatot.
	\1 1. lépés: ST (Specifikáció Tétele) miatt elég belátni: $Q \implies lf(S,R)$
	\1 2. lépés: fenti állítás belátása, $S$-től függően:
		\2 értékadás/kiválasztás: közvetlenül belátható ($R$-be behelyettesítés)
		\2 szekvencia: SZLSZ (Szekvencia Levezetési Szabálya) használata
		\2 elágazás: ELSZ (Elágazás Levezetési Szabálya) használata
		\2 ciklus: CLSZ (Ciklusz Levezetési Szabálya) használata
	\1 A 2. lépésben rekurzívan ismételjük a 2. lépést, amíg be nem látunk minden állítást.
\end{outline}

\subsection{Megvalósítási tippek}

\begin{outline}
	\1 Belátok valamit, ami nem definícióból következik (pl. $1<2$): PIPA
	\1 Értékkiválasztás: $lf(x:\in h, x \; even) = (h \ne \emptyset \wedge \forall e \in h : e \; even)$
	\1 $Q \implies ...$ esetén vegyem úgy, hogy $Q$ igaz és lássam be $...$-ot
	\1 ($\forall k \in [1..n]: x[k]=...$)-on ($x[i]=...$) végrehajtása: $\forall k$ 3 részre bontása: ($\forall k \in [1..i-1]: ...$) és ($x[i]=...$) és ($\forall k \in [i+1..n]: ...$)
	\1 $\lor$ (vagy)-ból következés: esetszétválasztás, akár negált is felhasználható, pl.: ($a \lor b \implies ...$)-nél: ($a \implies$) vagy ($\lnot a \wedge b \implies$)
	\1 Értékadások és számolások, amik abortálhatnak:
		\2 $a := b$ ahol $a : A$ és $b : B$ abortál, ha $B \not \subseteq A$
		\2 $a \mod b$ abortál, ha $b=0$
		\2 $x[i]$ ahol $x:\mathbb{Z}^n$ abortál, ha $i \notin [1..n]$
\end{outline}

\section{Pszeudokód jelölések}

\begin{outline}
	\1 Ciklus: $while \; VALAMI \; do \; [...] \; od$
\end{outline}

\pagebreak

\section{Párhuzamos programok}

\begin{outline}
	\1 Értékadás és feltétel kiértékelés nem megszakítható (atomikus)
\end{outline}

\subsection{Annotációk (pl. $\{Q_7\}$)}

\begin{outline}
	\1 Egy program egy adott pontjához egy logikai függvény: elő/utófeltétel.
	\1 Segédváltozókat is felvehetünk ennek érdekében. Ezzel a program megváltozik, de a helyessége nem: a segédváltozók nem módosítjak a nem-segédeket.
\end{outline}

\subsection{Atomi program}

\begin{outline}
	\1 $[S](a) = S(a)$
	\1 Atomikus: nem megszakítható
	\1 $S$ nem tartalmazhat sem ciklust, sem várakoztató utasítást (?)
	\1 Levezetési szabály:
		\2 Ha $Q \implies lf(S,R)$
		\2 Akkor $Q \implies lf([S],R)$
\end{outline}

\subsection{Várakoztató utasítás}

\begin{outline}
	\1 $\beta \in A \to \mathbb{L}$ (őrfeltétel)
	\1 Stukis jelölés: szekvencia, ahol a felső rész felet boltív van ($\cap$)
	\1 $(await \; \beta \; then \; S \; ta)(a) = $
		\2 ha $a \in D_\beta \wedge \beta(a)$ akkor $[S](a)$ \;\; ($\beta$ és $S$ együtt atomikus)
		\2 ha $a \in D_\beta \wedge \lnot \beta(a)$ akkor $(skip,await \; \beta \; then \; S \; ta)(a)$
		\2 ha $a \notin D_\beta$ akkor $<a,fail>$
	\1 Levezetési szabály
		\2 Ha
			\3 $Q \implies \beta \lor \lnot \beta$ \;\; ($\beta$ kiértékelhető)
			\3 $Q \wedge \beta \implies lf(S,R)$
		\2 Akkor $Q \implies lf(await \; \beta \; then \; S \; ta,R)$
	\1 $S$ nem tartalmazhat sem ciklust, sem várakoztató utasítást (?)
\end{outline}

\pagebreak

\subsection{Párhuzamos blokk}

\begin{outline}
	\1 $S_i$: komponens programok
		\2 vagy $S_i$ atomikus (nem megszakítható)
		\2 vagy $S_i=(u_i,T_i)$ ahol $u_i$ az első utasítás, $T_i$ a többi
	\1 Stukis jelölés: $x:=x+2 \; || \; x := 0$
	\1 $(parbegin \; S_1||...||S_n \; parend)(a)=\bigcup_{i=1}^n B_i(a)$
	\1 $B_i(a) = $
		\2 ha $S_i$ atomikus: $(S_i;parbegin \; S_1||...||S_{i-1}||S_{i+1}||...||S_n \; parend)(a)$
		\2 ha $S_i=(u_i,T_i)$: $(u_i;parbegin \; S_1||...||S_{i-1}||T_i||S_{i+1}||...||S_n \; parend)(a)$
	\1 Holtpontra jutott, ha minden be nem fejeződött komponense
		\2 vagy egy várakoztató utasítás, aminek őrfeltétele hamis (azaz blokkolt)
		\2 vagy egy szekvencia, amelyben van blokkolt várakoztató utasítás
	\1 Levezetési szabály: (PLSZ)
		\2 Legyenek $Q_i$ és $R_i$ ($i=[1..n]$) logikai függvények (annotációk)
		\2 Ha
			\3 $Q \implies Q_1 \wedge ... \wedge Q_n$ \;\; (belépési feltétel)
			\3 $R_1 \wedge ... \wedge R_n \implies R$ \;\; (jó helyen állunk meg)
			\3 $\forall i \in [1..n] : Q_i \implies lf(S_i,R_i)$
			\3 $Q_i \implies lf(S_i,R_i)$ teljes helyességi interferencia formulák interferenciamentesek
			\3 párhuzamos blokk holtpontmentes
		\2 Akkor $Q \implies lf(parbegin \; S_1||...||S_n \; parend, R)$
\end{outline}

\subsection{Kritikus utasítás}

\begin{outline}
	\1 Vagy egy értékadás
	\1 Vagy egy atomikusan végrehajtott program, ami tartalmaz értékadást
		\2 A várakoztató utasítás atomikus!!!
\end{outline}

\pagebreak

\subsection{Interferenciamentesség}

\begin{outline}
	\1 Legyen $u$ egy $pre_u$ előfeltételű kritikus utasítás $S_i$-ben
	\1 Legyen $s$ egy tetszőleges, $pre_s$ előfeltételű utasítás $S_j$-ben ($S_j$ tetszőleges)
	\1 $u$ nem interferál a $Q_j \implies lf(S_j,R_j)$ teljes helyességi formulával ha:
		\2 $pre_u \wedge R_j \implies lf(u,R_j)$ \;\; ($u$ végrehajtása nem rontja el $R_j$-t)
		\2 $pre_u \wedge pre_s \implies lf(u,pre_s)$ \;\; ($u$ végrehajtása nem rontja el $pre_s$-t)
		\2 $pre_u \wedge t=t_0 \implies lf(u, t \le t_0)$\\
		ahol $t$ egy tetszőleges $S_j$-beli ciklus terminálófüggvénye \\
		($u$ végrehajtása miatt nem "számol" visszafelé ciklus)
	\1 Gyakorlatban:
		\2 $Q_1 = (x=0)$; $Q_2=igaz$
		\2 $R_1=R_2 = (x=0) \lor (x=2)$
		\2 be kell látni: $1 \leadsto 2: \{Q_1\}x := x+2 \leadsto Q_2,R_2$
			\3 első rész: $R_2 \wedge Q_1 \implies lf(x:=x+2, R_2)$
			\3 második rész: $Q_2 \wedge Q_1 \implies lf(x := x+2, Q_2)$
		\2 be kell látni: $2 \leadsto 1: \{Q_2\}x := 0 \leadsto Q_1,R_1$
			\3 Ha 1 ciklus lenne: $\{Q_2\}x := 0 \leadsto Q_1,R_1,P$ és fenti $t$-s
			\3 Ha nincs közös változó: "nincs krit. ut. az 1. ágra nézve"
		\2 Alternatíva: "elég" belátni: a kritikus utasítás egy annotációt sem ront el,
		valamint a ciklus terminálófüggvény sem számol felfelé.
		\2 Emlékeztető: hamis bármit implikál ($\implies$ előtt ellentmondás)
\end{outline}

\pagebreak

\subsection{Elégséges feltétel holtpontmentességre}

\begin{outline}
	\1 Legyen $m$ $await$ és $n$ párhuzamos blokk egy $S$ szekvenciában
		\2 $A_m: await \; \beta_m \; then \; S_m \; ta$
		\2 $T_n : parbegin \; S^n_1||...||S^n_{n_n} \; parend$
	\1 $D(S)=[\bigvee_{j=1}^{m} (pre(A_j) \wedge \lnot \beta_j)] \lor [\bigvee_{k=1}^{n}D1(T_k)]$
		\2 vagy egy $await$ miatt várok
		\2 vagy egy párhuzamos blokk van holtpontban
	\1 $D1(T_k) = [\bigwedge_{j=1}^{n_k} (post(S_i^k) \lor D(S_i^k))] \wedge [\bigvee_{i=1}^{n_k}D(S_i^k)]$
		\2 az összes komponens már befejezedőtött vagy holtpontban van
		\2 és legalább egy komponens holtpontban van
	\1 Ha $D(S)=HAMIS$, akkor biztosan nincs holtpont. Egyébként nem mondunk semmit.
\end{outline}

\subsubsection{Holtpontmentesség: egy await egy kétágú párhuzamos blokkban}

\begin{outline}
	\1 Legyen $S=parbegin \; (await \; \beta \; then \; S_1 \; ta) \; ||\; S_2 \; parend$
	\1 Bizonyítás nélkül használható tény: csak akkor van ekkor holtpont ha\\
	$pre(await \; \beta \; then \; S_1 \; ta) \wedge \lnot \beta \wedge post(S_2)$
	\1 Ha ez ellentmondást ad ($=hamis$), akkor nincs holtpont
\end{outline}

\subsubsection{Holtpontmentesség: két await egy kétágú párhuzamos blokkban}

\begin{outline}
	\1 Legyen $S=parbegin \; (await \; \beta_1 \; then \; S_1 \; ta) \; ||\; (await \; \beta_2 \; then \; S_2 \; ta) \; parend$
		\2 Átfogalamzva: $S=parbegin \; A_1||A_2 \; parend$
	\1 Bizonyítás nélkül használható tény: csak akkor van ekkor holtpont ha
		\2 vagy: $pre(A_1) \wedge \lnot \beta_1 \wedge post(A_2)$
		\2 vagy: $pre(A_2) \wedge \lnot \beta_2 \wedge post(A_1)$
		\2 vagy: $pre(A_1) \wedge \lnot \beta_1 \wedge pre(A_2) \wedge \lnot \beta_2$
	\1 Ha ez ellentmondást ad ($=hamis$), akkor nincs holtpont
\end{outline}

\subsubsection{Holtpontmentesség: nincs await}

\begin{outline}
	\1 Mindig holtpontmentes.
\end{outline}

\end{document}
